%%%%%%%%%%%%%%%%%%%%%%%%%%%%%%%%%%%%%%%%%%%%%%%%%%%%%%%%%%%%%%%%%%%%%%%%%%%%%%%%
%2345678901234567890123456789012345678901234567890123456789012345678901234567890
%        1         2         3         4         5         6         7         8

\documentclass[letterpaper, 10 pt, conference]{ieeetran}   



                                                          
%images
\usepackage{amsmath}
\usepackage{verbatim}
%\usepackage{graphicx}
\usepackage{subfig}
\usepackage{graphicx}
\usepackage{epstopdf}
\usepackage{amssymb}
\usepackage{float}
\usepackage{wrapfig}
\graphicspath{{images/}}
\usepackage{cite}
\usepackage[hidelinks]{hyperref}
\usepackage{adjustbox}
\usepackage{multirow}
\newtheorem{rem}{Remark}
\newtheorem{assumption}{Assumption}
\newtheorem{thm}{Theorem}
\newtheorem{pf}{Proof}
\newtheorem{lemma}{Lemma}
\usepackage{hyperref} 
%block diagram
\usepackage{tikz}
\usetikzlibrary{shapes,arrows}
\pagestyle{empty}
%block diagram layers
\pgfdeclarelayer{background}
\pgfdeclarelayer{foreground}
\pgfsetlayers{background,main,foreground}
\tikzstyle{int}=[draw, fill=blue!20, minimum size=2em]
\tikzstyle{init} = [pin edge={to-,thin,black}]
%block diagram
\usepackage{tikz}
\usetikzlibrary{shapes,arrows}
\pagestyle{empty}
\usetikzlibrary{calc,arrows}
\IEEEoverridecommandlockouts









\title{\LARGE \bf
Title
}


\author{A, B, and Cheng-Wei Chen % <-this % stops a space
\thanks{This work was financially supported from the Young Scholar Fellowship Program by Ministry of Science and Technology (MOST) in Taiwan, under Grant MOST 107-2636-E-002-008.}
\thanks{A is with Electrical Engineering, National Taiwan University, Taipei, Taiwan
        {\tt\small a@ntu.edu.tw}}%
\thanks{B is with Electrical Engineering, National Taiwan University, Taipei, Taiwan
        {\tt\small b@ntu.edu.tw}}%
\thanks{Cheng-Wei Chen is with Electrical Engineering, National Taiwan University, Taipei, Taiwan
        {\tt\small cwchenee@ntu.edu.tw}}%
}


\begin{document}
%\pagestyle{empty}

%block diagram
\tikzstyle{block} = [draw, fill=blue!20, rectangle, 
    minimum height=3em, minimum width=3em]
\tikzstyle{sum} = [draw, fill=blue!20, circle, node distance=1cm]
\tikzstyle{input} = [coordinate]
\tikzstyle{output} = [coordinate]
\tikzstyle{pinstyle} = [pin edge={to-,thin,black}]

\maketitle
\thispagestyle{empty}
\pagestyle{empty}
\maketitle


%%%%%%%%%%%%%%%%%%%%%%%%%%%%%%%%%%%%%%%%%%%%%%%%%%%%%%%%%%%%%%%%%%%%%%
% citation label: \cite{XXX}
% refer Fig or Table: \ref{XXX}
% add vertical space: \vspace{-5 mm}
% paragraph align to the most left edge: \noindent

% figure
%% \begin{figure}[h]
%% \begin{center}
%% \includegraphics[width=3.5cm]{DualStageScheme}
%% \caption{Dual-stage system scheme.}
%% \label{DualStageScheme}
%% \end{center}
%% \end{figure}

% subfigures
%% \begin{figure}[H]
%% 	\begin{center}
%% 		\subfloat[Master-slave configuration]{\adjincludegraphics[trim=0cm 0cm 0cm 0cm,  clip=true,width=8.5cm]{DualStageTPRRC2}}\\
%% 		\vspace{0 mm}
%% 		\subfloat[TPRRC]{\adjincludegraphics[trim=0cm 0cm 0cm 0cm, clip=true,width=5cm]{TPRRC}}\\
%% 		\caption{Block diagram of the master-slave dual-stage system using the two parameter robust repetitive control (TPRRC).}
%% 		\label{DualStageTPRRC}
%% 	\end{center}
%% \end{figure} 

% equation
%% \begin{equation}
%% P_{i}=\frac{C_{stab[i]}G_{i}}{1+C_{stab[i]}G_{i}},  i=1,2
%% \label{ClosedLoopplant}
%% \end{equation}

% multiple equations
%% \begin{align}
%% \mathbf{r}&\equiv\begin{bmatrix}
%% \mathbf{y_i}&\equiv\begin{bmatrix}
%% \mathit{y_i}(t)  & \mathit{y_i}(t+\Delta t)   &  ...  & \mathit{y_i}(t+(n-1)\Delta t) 
%% \end{bmatrix}^T \label{OutputVec}\\
%% \mathbf{e_i}&\equiv\begin{bmatrix}
%% \mathit{e_i}(t)  & \mathit{e_i}(t+\Delta t)   &  ...  & \mathit{e_i}(t+(n-1)\Delta t) 
%% \end{bmatrix}^T \label{ErrVec}
%% \end{align}
%%%%%%%%%%%%%%%%%%%%%%%%%%%%%%%%%%%%%%%%%%%%%%%%%%%%%%%%%%%%%%%%%%%%%%
\begin{abstract}
Here is the abstract    


\end{abstract}

%=================================================================% 
%===========================SEC 1=================================% 
%===========================INTRO=================================% 
%=================================================================% 

\section{Introduction}
\label{sec:intro}

Here is the introduction \cite{chen2020iterative}, as shown in Eq. \ref{eq:Ztransform}.

\begin{equation}
G(z):=\sum_{k=-N/2}^{N/2}g(k)z^{-k}
\label{eq:Ztransform}
\end{equation}


The main contributions of this paper are: 

\begin{enumerate}
	\item one;
	\item two;
	\item threes.
\end{enumerate}

\begin{figure}[h]
 	\begin{center}
 	\includegraphics[width=8cm]{ILC.eps}
 	\caption{The generic ILC scheme.}
 	\label{fig:ILC}
 	\end{center}
\end{figure}



 
%=================================================================% 
%===========================SEC 5=================================% 
%===========================CONCLUSION============================% 
%=================================================================% 
 
\section{Conclusion}
\label{sec:conclu}  
Here is the conclusion.


%=================================================================%  
%==========================REFERENCE==============================% 
%=================================================================% 
\bibliographystyle{IEEEtr}
\bibliography{refbib}




%=================================================================% 
\end{document}
